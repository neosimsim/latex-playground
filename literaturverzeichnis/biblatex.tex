% The MIT License (MIT)
% 
% Copyright (c) 2015 Alexander Ben Nasrallah
% 
% Permission is hereby granted, free of charge, to any person obtaining a copy
% of this software and associated documentation files (the "Software"), to deal
% in the Software without restriction, including without limitation the rights
% to use, copy, modify, merge, publish, distribute, sublicense, and/or sell
% copies of the Software, and to permit persons to whom the Software is
% furnished to do so, subject to the following conditions:
% 
% The above copyright notice and this permission notice shall be included in all
% copies or substantial portions of the Software.
% 
% THE SOFTWARE IS PROVIDED "AS IS", WITHOUT WARRANTY OF ANY KIND, EXPRESS OR
% IMPLIED, INCLUDING BUT NOT LIMITED TO THE WARRANTIES OF MERCHANTABILITY,
% FITNESS FOR A PARTICULAR PURPOSE AND NONINFRINGEMENT. IN NO EVENT SHALL THE
% AUTHORS OR COPYRIGHT HOLDERS BE LIABLE FOR ANY CLAIM, DAMAGES OR OTHER
% LIABILITY, WHETHER IN AN ACTION OF CONTRACT, TORT OR OTHERWISE, ARISING FROM,
% OUT OF OR IN CONNECTION WITH THE SOFTWARE OR THE USE OR OTHER DEALINGS IN THE
% SOFTWARE.

\documentclass[a4paper]{article}

\usepackage[utf8]{inputenc}
\usepackage[ngerman]{babel}
\usepackage{listings}
\usepackage[colorlinks=true, linktoc=page]{hyperref}
\usepackage[]{color}
\usepackage[backend=biber]{biblatex}                 % Biber ist eigentlich der Standardwert. Eine Alternative ist bibtex.
\addbibresource{literatur.bib}

\begin{document}
Es gibt verschiedene Varianten ein Literaturverzeichnis mit \LaTeX{} zu
erstellen.

\lstdefinestyle{latex}{
	keywordstyle=\bfseries,
	backgroundcolor=\color{white},
	frame=single,
}

\lstset{
	language=TeX,
	morekeywords={begin,bibitem,emph,addbibresource},
	tabsize=4,
	style=latex
}

\tableofcontents

\section{BibLaTeX}
\emph{BibLaTeX} ist der Nachfolger von \emph{BibTeX} und wir werden uns hier auf
\emph{BibLaTeX} einschrenken. \emph{BibLaTeX} verwendet eine Datenbank zu
referenzieren. Dies hat den Vorteil, dass man leichter aus mehreren .tex-Dateien
referenzieren kann.
\lstinputlisting[caption={literatur.bib}]{literatur.bib}

Die Verwendung ist allerdings aufwendiger als mit der
\emph{thebibliography}-Umgebung.
\begin{enumerate}
	\item Das Paket \emph{biblatex} muss geladen werden.
	\item Die Datenbankdatei musst angegeben werden
		\begin{lstlisting}
\addbibresource{literatur.bib}
		\end{lstlisting}
	\item Nun kann man mit \verb+\cite{greenwade93}+ zitieren.
	\item Nun wird \LaTeX{} beim Kompilieren einen Fehler melden, das die Referenzen nicht bekannt sind. Daher sind folgende Schritte beim für ein erfolgreiches Kompilieren notwendig.
		\begin{lstlisting}[language=sh, morekeywords={latex,biber}]
latex datei.tex
biber datei.bcf || bibtex datei.aux
latex datei.tex
latex datei.tex
		\end{lstlisting}
\end{enumerate}
\cite{greenwade93}

\printbibliography

\url{http://en.wikibooks.org/wiki/LaTeX/Bibliography_Management}
\url{http://literatur-generator.de/}
\end{document}


