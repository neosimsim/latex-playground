% The MIT License (MIT)
% 
% Copyright (c) 2015 Alexander Ben Nasrallah
% 
% Permission is hereby granted, free of charge, to any person obtaining a copy
% of this software and associated documentation files (the "Software"), to deal
% in the Software without restriction, including without limitation the rights
% to use, copy, modify, merge, publish, distribute, sublicense, and/or sell
% copies of the Software, and to permit persons to whom the Software is
% furnished to do so, subject to the following conditions:
% 
% The above copyright notice and this permission notice shall be included in all
% copies or substantial portions of the Software.
% 
% THE SOFTWARE IS PROVIDED "AS IS", WITHOUT WARRANTY OF ANY KIND, EXPRESS OR
% IMPLIED, INCLUDING BUT NOT LIMITED TO THE WARRANTIES OF MERCHANTABILITY,
% FITNESS FOR A PARTICULAR PURPOSE AND NONINFRINGEMENT. IN NO EVENT SHALL THE
% AUTHORS OR COPYRIGHT HOLDERS BE LIABLE FOR ANY CLAIM, DAMAGES OR OTHER
% LIABILITY, WHETHER IN AN ACTION OF CONTRACT, TORT OR OTHERWISE, ARISING FROM,
% OUT OF OR IN CONNECTION WITH THE SOFTWARE OR THE USE OR OTHER DEALINGS IN THE
% SOFTWARE.

\documentclass[a4paper]{article}

\usepackage{ngerman}
\usepackage{eurosym}
\usepackage{units}
\usepackage{listings}

\title{Typografies\"unden}
\author{Alexander Ben Nasrallah}

\begin{document}
\maketitle
\begin{abstract}
	Dieses Dokument gibt eine \"Ubersicht an typografischen Konventionen und
	erkl\"art, wie man sie in \LaTeX{} umsetzt.

	Wir richten uns hierbei an die deutsche Sprache.
\end{abstract}

\begin{description}
	\item[Abstand vor Einheiten] Im Deutschen ist es \"ublich zwischen zwischen
		Zahl und Ma\ss{}einheit bzw. Symbol steht ein Leerzeichen zu lassen. Die
		einzige Ausnahme ist der Winkelgrad. \LaTeX{} liefert sogar die Option
		ein \emph{halbes} und \emph{gesch\"utztes} Leerzeichen mit \verb+\,+ zu
		setzen. Das Paket \emph{units} nimmt einem die Aufgabe ab, auf solche
		Abst\"ande zu achten.

		\begin{tabular}[c]{|rl|}
			\hline
			\bf{Beispiel}             & \bf{\LaTeX}                       \\
			\hline
			25\,\euro                 & \verb+25\,\euro+                  \\
			\unit[80]{\%}             & \verb+\unit[80]{\%}+              \\
			$45^\circ$                & \verb+$45^\circ$+                 \\
			$100\,^{\circ}\mathrm{C}$ & \verb+$100\,^{\circ}\mathrm{C}$+ \\
			\hline
		\end{tabular}
	\item[Abstand bei K\"urzeln] Die Regel, dass nach einem Punkt ein
		Leerschritt zu folgen hat, gilt auch für Abkürzungen. Auch hier gilt,
		dass der Abstand aus optischen Gründen auf halbe Laufweite reduziert
		werden kann. Hier sollte man darauf achten, ein gesch\"utztes
		Leerzeichen zu verwenden, sodass \LaTeX{} keinen Zeilenumbruch
		einf\"ugt.

		\begin{tabular}[c]{|rl|}
			\hline
			\bf{Beispiele} & \bf{\LaTeX}       \\
			\hline
			u.\,v.\,m.     & \verb+u.\,v.\,m.+ \\
			d.\,h.         & \verb+d.\,h.+     \\
			\hline
		\end{tabular}

\end{description}
\end{document}


